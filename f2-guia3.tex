%% LyX 2.0.3 created this file.  For more info, see http://www.lyx.org/.
%% Do not edit unless you really know what you are doing.
\documentclass[11pt,spanish]{article}
\usepackage{mathptmx}
\usepackage[T1]{fontenc}
\usepackage[latin1]{inputenc}
\usepackage[a4paper]{geometry}
\geometry{verbose,tmargin=2cm,bmargin=2cm,lmargin=2cm,rmargin=2cm}
\usepackage{float}
\usepackage{graphicx}

\makeatletter
%%%%%%%%%%%%%%%%%%%%%%%%%%%%%% User specified LaTeX commands.
%%\usepackage{SEART}

%TCIDATA{TCIstyle=article/art4.lat,SEART,SEART}

%TCIDATA{Created=Mon Aug 20 21:09:02 2001}
%TCIDATA{LastRevised=Tue May 14 08:11:44 2002}
%% \input{tcilatex}

\makeatother

\usepackage{babel}
\addto\shorthandsspanish{\spanishdeactivate{~<>}}

\begin{document}
\begin{center}
\textsc{\large Física 2 (Físicos) - Cátedra Diana Skigin}
\par\end{center}{\large \par}

\begin{center}
\textsc{\large Segundo Cuatrimestre de 2020}
\par\end{center}{\large \par}

\begin{center}
\textsc{\large Guía 3: Descripción geométrica de movimientos ondulatorios}
\par\end{center}{\large \par}
\begin{enumerate}
\item 
\begin{enumerate}
\item Si un rayo parte del punto $A=(0,1,0)$, se refleja en el espejo plano
$(x,0,z)$ y pasa por el punto $B=(4,3,0)$, averigüe en qué punto
sobre el plano del espejo se refleja y los ángulos de incidencia y
reflexión. Aplicar Fermat e interpretar físicamente las soluciones. 
\item Un rayo directo entre $A$ y $B$ recorre un menor camino óptico que
el hallado en (a), ¿es esto contradictorio?
\end{enumerate}
\item A partir del principio de Fermat deducir la ley de Snell
para la refracción de la luz entre dos medios de índices
 $n_{1}$ y $n_{2}$, separados por una superficie plana.
\item Sea un espejo elíptico de focos $A$ y $B$. En $A$ hay una fuente
puntual. Los espejos esférico y plano dibujados son tangentes al elíptico
en $C$. Sabiendo que el camino óptico de un rayo que sale de $A$,
se refleja en $C$ y luego pasa por $B$, es estacionario en la elipse,
obtenga cualitativamente si el camino óptico es máximo, mínimo o estacionario
cuando se refleja en cada uno de los espejos.
\item 
\begin{enumerate}
\item Demuestre que un rayo que incide sobre una lámina de caras paralelas,
inmersa en un medio único, no se desvía al atravesarla. Calcule el
desplazamiento lateral de dicho rayo, en términos de su espesor $d$
y de su índice de refracción $n$.
\item Demuestre que el rayo que se refleja en la primera cara y el que emerge
luego de reflejarse en la segunda son paralelos.
\item Si el medio exterior es único, ¿existe algún ángulo de incidencia
tal que produzca reflexión total en la cara inferior?
\end{enumerate}
\item Un rayo incide con ángulo $\phi$ sobre la superficie horizontal de
un cubo de material transparente, de índice $n$, inmerso en aire.
\begin{figure}[H]
\centering{}\includegraphics[clip,scale=0.25]{ej3-5}
\end{figure}

\begin{enumerate}
\item ¿Para qué valores de $\phi$ hay reflexión total en la cara vertical?
\item Si $\phi=60^{\circ}$, ¿cuál es el máximo $n$ para que no haya reflexión
total en la cara vertical? ¿Se puede reflejar totalmente en la cara
superior?
\end{enumerate}
\item 
\begin{enumerate}
\item Calcule analíticamente el ángulo de desviación mínima del prisma.
Justifique por qué este valor es único. Haga un gráfico cualitativo
de la desviación como función del ángulo de incidencia.
\item Calcule la desviación mínima para prismas delgados, en función de
los datos constructivos.
\item Si el prisma es delgado y el ángulo de incidencia es pequeño, calcule
la desviación.
\end{enumerate}
\item Los índices de refracción de cierta clase de vidrio para el rojo y
el violeta valen: $1.51$ y $1.53$; respectivamente. Halle los ángulos
límites de reflexión total para rayos que incidan en la superficie
de separación vidrio-aire. ¿Qué ocurre si un rayo de luz blanca incide
formando un ángulo de 41$^{\circ}$ sobre dicha superficie?
\item 
\begin{enumerate}
\item En un vidrio óptico común se propaga un haz de luz blanca, ¿qué componente
viaja más rápido: la roja o la violeta?
\item ¿Para cuál de ambos colores será mayor la desviación en un prisma?
¿Qué puede decir del ángulo de desviación mínima? Justifique sus respuestas.
\end{enumerate}
\item Dado un prisma de Crown de ángulo $\alpha=4^{\circ}$ calcular, para
las líneas F, D y C, las desviaciones de rayos que inciden casi perpendicularmente.
Los respectivos índices son: $n_{F}=1.513$; $n_{D}=1.508$ y $n_{C}=1.504$.
\item 
\begin{enumerate}
\item Demuestre que la imagen dada por un espejo plano de una fuente puntual
es, sin ninguna aproximación, otra fuente puntual, ubicada simétricamente
respecto del plano del espejo. Analice los casos que corresponden
a objetos reales o virtuales.
\item ¿Cuál es la mínima longitud de un espejo plano vertical para que un
hombre de $1.8$ m se vea entero? ¿Es importante conocer la distancia
hombre-espejo? 
\end{enumerate}
\item Haga un esquema de un diagrama de rayos localizando las imágenes de
la flecha que se muestra en la figura. Para un punto de la flecha
dibuje una porción del frente de ondas emergente y los correspondientes
frentes reflejados.
\begin{figure}[H]
\centering{}\includegraphics[clip,scale=0.25]{ej3-11}
\end{figure}

\item (*) Dos espejos planos forman un ángulo $\alpha$ como lo indica la figura.
\begin{figure}[H]
\centering{}\includegraphics[clip,scale=0.25]{ej3-12}
\end{figure}
\begin{enumerate}
\item Un rayo de luz contenido en un plano perpendicular a la intersección
de los espejos incide sobre uno de ellos, se refleja e incide en el
otro (ver figura). Calcule el ángulo que forman los rayos incidente
y emergente.
\item Suponga la misma geometría que en (a) pero ahora iluminada por una
fuente puntual, demuestre que las imágenes se encuentran sobre una
circunferencia con centro en el vértice de los espejos. En el caso
en que la fuente está ubicada de tal modo que sólo se producen dos
imágenes, y que el ángulo es muy pequeño, calcule la distancia entre
ellas (espejos de Fresnel).
\end{enumerate}
\item 
\begin{enumerate}
\item Demostrar que un haz homocéntrico de pequeña abertura que incide casi
normal sobre una dioptra plana, da lugar a otro haz homocéntrico.
Considere los casos de objetos reales y virtuales.
\item Una moneda se encuentra en el fondo de un vaso que contiene agua hasta
una altura de 5 cm ($n_{agua}=1.33$). Un observador la mira desde
arriba, ¿a qué profundidad la ve?
\item Estimar la máxima abertura de un haz homocéntrico, para que la posición
de la imagen, formada por una única superficie plana, quede determinada
con un error del 2\%. 
\end{enumerate}
\item Usando los resultados del problema anterior demuestre que un haz homocéntrico
de pequeña abertura, al atravesar una lámina de caras paralelas, da
lugar, en primera aproximación, a otro haz homocéntrico. Halle la
posición de las sucesivas imágenes. 
\begin{figure}[H]
\centering{}\includegraphics[clip,scale=0.25]{ej3-15}
\end{figure}
\item Haciendo uso de la figura, de la ley de Snell y del hecho de que en
la aproximación paraxial $\alpha\approx\sin\alpha\approx\tan\alpha$
(lo mismo pasa con $\beta$ y $\varphi$) obtenga la ecuación de las
dioptras esféricas, que establece lo siguiente:
\[
\frac{n'}{s'}\mp\frac{n}{s}=\frac{(n'-n)}{R}
\]
Discuta el doble signo, asociándolo con la convención de signos que
se utilice.
\item Para una dioptra esférica arbitraria haga un gráfico $s'$ vs $s$
y analice a partir de él para qué posiciones de los objetos reales
las imágenes son reales o virtuales, directas o invertidas y lo mismo
para objetos virtuales. Analice todos los casos posibles para dioptras
convergentes y divergentes.
\item ¿Pueden ser iguales las dos distancias focales de una dioptra?. Justifique
su respuesta.
\item La esfera de vidrio de la figura, de 1cm de diámetro, contiene una
pequeña burbuja de aire desplazada $0.5$ cm de su centro. Hallar
la posición y el aumento de la burbuja cuando se la observa desde
\emph{A} y cuando se la observa desde \emph{B}.
\begin{figure}[H]
\centering{}\includegraphics[clip,scale=0.25]{ej3-17}
\end{figure}

\item 
\begin{enumerate}
\item Partiendo de la ecuación de las dioptras obtenga la ecuación de los
espejos esféricos. 
\item ¿Cómo se modifica la distancia focal de un espejo esférico si se lo
sumerge en agua?
\item Un espejo esférico cóncavo produce una imagen cuyo tamaño es el doble
del tamaño del objeto, siendo la distancia objeto--imagen de 15 cm.
Calcule la distancia focal del espejo.
\end{enumerate}
\item 
\begin{enumerate}
\item A partir de la ecuación de la dioptra, considerando dos dioptras esféricas
tal que la separación entre ellas sea mucho menor que las restantes
longitudes involucradas, deduzca la ecuación para las lentes delgadas.
\item Analice de qué depende la convergencia o divergencia de una lente.
\item Grafique $s'$ vs $s$ para lentes convergentes y divergentes, analice
el aumento y la posición de los objetos (en particular objeto en el
foco y objeto en infinito) y de las imágenes.
\item ¿Pueden ser iguales (en módulo) los focos de una lente?
\item Demuestre que la menor distancia objeto--imagen es $4f$, si la lente
está inmersa en un único medio.
\item Dibuje los frentes de onda incidente, refractado por la primer dioptra
y refractado por la segunda.
\end{enumerate}
\item 
\begin{enumerate}
\item Determine la distancia focal de una lente plano--cóncava ($n=1,5$)
cuyo radio de curvatura es 10 cm. Determine su potencia en dioptrías.
\item Se tiene una lente biconvexa con $R_{1}=R_{2}=$10 cm, construida
con un vidrio de índice $1.5$. Se la usa con aire a un lado de la
misma y con un líquido de índice $1.7$ al otro lado. ¿Cuánto valen
las distancias focales? ¿Es convergente o divergente? Responda las
mismas preguntas si: i) está inmersa sólo en aire, ii) está inmersa
en el medio de índice $1.7$.
\end{enumerate}

\begin{enumerate}
\item Se coloca un objeto a 18 cm de una pantalla, ¿en qué puntos entre
la pantalla y el objeto se puede colocar una lente delgada convergente
de distancia focal 4 cm, para que la imagen del objeto esté sobre
la pantalla? ¿Qué diferencia hay entre ubicarla en una u otra posición?
\item Un objeto se halla a distancia fija de la pantalla. Una lente delgada
convergente, de distancia focal 16 cm, produce imagen nítida sobre
la pantalla cuando se encuentra en dos posiciones que distan entre
sí 60 cm. ¿Cuál es la distancia objeto--pantalla?
\end{enumerate}
\item Halle la distancia focal de una lente sumergida en agua, sabiendo
que su distancia focal en el aire es de 20 cm. El índice de refracción
del vidrio de la lente es $1.6$. 
\item Se tiene una lente delgada en las condiciones que presenta la figura.
Indique en qué punto del eje óptico debe incidir un rayo para que
atraviese la lente sin desviarse. Exprese el resultado en función
de la distancia focal objeto y de los índices de refracción.
\begin{figure}[H]
\centering{}\includegraphics[clip,scale=0.25]{ej3-25}
\end{figure}
\item 
\begin{enumerate}
\item Determine el radio de curvatura de una lupa equiconvexa ($n=1,5$)
para que su aumento sea 10X. ¿Dónde se encuentra la imagen, y el objeto?
\item Calcule el aumento de la lupa descripta en (a) cuando la imagen se
encuentra a la distancia de visión clara. Discuta las ventajas y desventajas
de esta opción.
\end{enumerate}
\item Una lente delgada convergente, de distancia focal 30 cm, se coloca
20 cm a la izquierda de otra lente delgada divergente de distancia
focal 50 cm. Para un objeto colocado a 40 cm a la izquierda de la
primera lente determine la imagen final. ¿Cuál es el aumento? La imagen
¿es real o virtual, es directa o invertida?
\item Un microscopio consta de un objetivo de 4 mm de distancia focal y
de un ocular de 30 mm de distancia focal. La distancia entre el foco
imagen del objetivo y el foco objeto del ocular es $g=$18 cm. Calcule:

\begin{enumerate}
\item El aumento normal del microscopio.
\item La distancia objeto--objetivo.
\item Sabiendo que el microscopio no cuenta con diafragmas adicionales,
y que la pupila de salida debe ser real, y del mismo diámetro aproximado
que la pupila del ojo ($\approx$12 mm), discuta cuál de las dos lentes
debe ser el diafragma de apertura, cuál debe ser su diámetro y en
qué posición se halla la pupila de salida.
\item Discuta en qué posiciones colocaría un diafragma de campo, y si esta
introducción modifica o no la determinación del diafragma de apertura.
\end{enumerate}
\vspace{3mm}
\item Un anteojo astronómico utiliza como objetivo una lente convergente
de 2 m de distancia focal y 10 cm de diámetro, y como ocular una lente
convergente de 4 cm de distancia focal. Determine:
\begin{enumerate}
\item El aumento eficaz.
\item Las características de la primer imagen de la luna y de la imagen
final a través del telescopio. La luna subtiende, a ojo desnudo, un
ángulo de 31'.
\item El largo total del tubo.
\item El mínimo diámetro del ocular para que el objetivo sea diafragma de
apertura. (Recordar que la luna no es puntual, y por ende hay puntos
objeto extra-axiales).
\item Suponiendo que el diámetro del ocular sea de 4 cm, la posición y el
tamaño de la pupila de salida (*).
\item La posición en que debe colocarse el ojo.
\item La posición en que debe colocarse, de ser posible, un diafragma de
campo.
\item El mínimo diámetro del posible diafragma de campo para que la imagen
de la luna se vea completa 
\end{enumerate}
\end{enumerate}

\end{document}
